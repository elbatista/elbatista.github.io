\documentclass[11pt]{article}
\usepackage[a4paper,margin=1in]{geometry}
\usepackage{parskip}
\usepackage{hyperref}

\begin{document}

\noindent
\textbf{Eliã Rafael de Lima Batista} \\
\href{mailto:o.elia.batista@gmail.com}{o.elia.batista@gmail.com} / \href{mailto:delime@usi.ch}{delime@usi.ch} \\
+41 77 992 5544 \\
\href{https://linkedin.com/in/e-batista}{linkedin.com/in/e-batista} \\
\href{https://elbatista.github.io}{elbatista.github.io}

\vspace{1em}

Lugano, 22 ottobre 2025 \\
\vspace{1em}

Alla cortese attenzione del Team di Selezione \\
\textbf{Azienda Elettrica di Massagno (AEM)}

% \vspace{1em}

Oggetto: Candidatura per la posizione di \textbf{Full Stack Developer}

\vspace{1em}

Gentili Signore, Egregi Signori,

\vspace{1em}

Desidero sottoporre la mia candidatura per la posizione di \textbf{Full Stack Developer} presso l’Azienda Elettrica di Massagno (AEM). Sono un ingegnere informatico con oltre dieci anni di esperienza nello sviluppo di applicazioni web e mobile, e attualmente sto conseguendo un \textbf{Dottorato in Informatica presso l’Università della Svizzera italiana (USI)}, con ricerca nell’ambito dei sistemi distribuiti e ad alta affidabilità.

Nel corso della mia carriera ho lavorato su progetti complessi che combinano sviluppo front-end, back-end e architetture scalabili. Presso \textit{Compasso UOL}, ho sviluppato applicazioni full stack e mobile utilizzando tecnologie web (HTML, CSS, JavaScript), frameworks \textbf{React}, \textbf{React Native} e \textbf{Node.js}, collaborando strettamente con designer e stakeholder per garantire soluzioni di alta qualità orientate all’utente. Presso la stessa azienda, ho avuto modo di lavorare anche con la progettazione di interfacce \textbf{UI/UX}, con pipeline \textbf{CI/CD} e con tecnologie di containerizzazione come \textbf{Docker}.

La mia attività di ricerca accademica integra la mia esperienza industriale, rafforzando il mio approccio orientato alla qualità del codice, alle prestazioni e all’affidabilità del software. Lavoro quotidianamente in ambienti Linux e utilizzo linguaggi come \textbf{Java}, \textbf{Go} e \textbf{Python} per progettare sistemi resilienti e performanti.

Ciò che mi motiva particolarmente di AEM è la possibilità di contribuire allo sviluppo di soluzioni digitali innovative nel settore energetico, un ambito che considero strategico per il futuro. L’approccio interdisciplinare di AEM, che combina analitiche avanzate, automazione e smart grid, rappresenta per me l’ambiente ideale per mettere a frutto le mie competenze tecniche e la mia passione per l’innovazione.

Vivo a Lugano da cinque anni e sono in possesso di un \textbf{permesso di soggiorno valido che mi consente di lavorare in Svizzera}. Il permesso è rinnovato annualmente ed è legato a un contratto di lavoro. Essendo residente nel Canton Ticino, sono pienamente disponibile a collaborare in presenza presso la sede di AEM.

% \vspace{.5em}

Ringrazio sinceramente per l’attenzione dedicata alla mia candidatura. Sarei lieto di poter contribuire con la mia esperienza e il mio entusiasmo alle attività dell’unità “Innovazione e Soluzioni Digitali” di AEM, collaborando allo sviluppo di nuovi servizi energetici intelligenti in Ticino.

\vspace{1em}

Distinti saluti, \\
\textbf{Eliã Rafael de Lima Batista}

\end{document}
